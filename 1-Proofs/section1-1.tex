\documentclass{article}
\usepackage{amsmath,amssymb,amsthm}
\usepackage[colorlinks=true,urlcolor=blue]{hyperref}
\usepackage[a4paper, total={7in,10in}]{geometry}

\begin{document}
\section{What is a Proof?}
\subsection{Problem 1.1}
Can't easily do a visual prolem in \LaTeX, and I am too lazy to create graphics for this. I will try to use my words instead.
\subsubsection{Problem 1.1 (a)}
Arrange the triangles so that the ``a'' side of one is on the ``b'' side of the other, with the acute angles touching. Do this for all of them, and the ``c'' sides will form an outer square of size $c\times c$, and the square hole in the middle will be of size $(b-a)\times(b-a)$.
\subsubsection{Problem 1.1 (b)}
Move two adjacent triangles to the opposite side of the $c\times c$ sqaure, so that their ``c'' sides are flush, forming two $a\times b$ rectangles, touching each other and the $(b-a)\times(b-a)$ square. This forms a single shape (that looks like a 'P' turned $90^{\circ}$ to the right. They form two squares of size $a\times a$ and $b\times b$. The $b\times b$ square includes the $(b-a)\times(b-a)$ square, one of the $b\times a$ rectangles, and a small section of the other one. The $a\times a$ square is formed with the rest of that rectangle.
\subsubsection{Problem 1.1 (c)}
In the case of $a=b$, The $(b-a)\times(b-a)$ rectangle becomes a degenerate $0\times0$ rectangle. However this does not pose a problem the proof still holds. As $a=b$, the triangles are right-angle isoceles triangles. For (a), the triangles form a square with the sides forming an ``$\times$'' in the middle. For part (b), the triangles are rearranged into a $a\times2a$ rectangle, made up of two $a\times a$ squares, which shows that $a^2 + a^2 = a^2 + b^2 = c^2$ using the preservation of area under rearrangement.
\subsubsection{Problem 1.1 (d)}
1. That the two acute angles of a right-angled triangle sum to $90^{\circ}$.\\
2. When putting the right angle of a right-angled triangle against a straight line, they form a retroflex right angle (For part (a), these become the corners of the square).\\
3. That if you put a line of length $a$ along a line of length $b$, the remaining length not covered by the first line is of length $b-a$.\\
4. That is a shape has 4 right-angles and equal sides, it is a square. (I am not as confident in this one, because, isn't this the defintion of a square??)\\
5. That the acute angles from two right-angle triangles together sum to $180^{\circ}$.
\subsection{Problem 1.2}
\subsubsection{Problem 1.2 (a)}
$\sqrt{(-1)(-1)}=\sqrt{-1}\sqrt{-1}$ is not true.
\subsubsection{Problem 1.2 (b)}
Assume $1=-1$.
\begin{align}
  1 &= -1 \\
  \implies 2 &= 0 \\
  \implies 1 &= 0 \\
  \implies 2 &= 1
\end{align}
\subsubsection{Problem 1.2 (c)}
$$\sqrt{rs} = (rs)^{\frac{1}{2}} = r^{\frac{1}{2}}s^{\frac{1}{2}} = \sqrt{r}\sqrt{s}$$
\subsection{Problem 1.3}
\subsubsection{Problem 1.3 (a)}
$\log_{10}{(1/2)}$ is negative, and thus you cannot multiply both sides of the inequality without flipping the ``$>$'' into a ``$<$''.
\subsubsection{Problem 1.3 (b)}
``$\$$'' and ``¢'' should be treated as units. As $\$1=\$=100\text{¢}$, the first step is valid. However, the step $\$0.01 = (\$0.1)^2$ is invalid because the RHS expands to $\$^20.01$, which is not equal to $\$0.01$ as $\$^2\ne\$$. Similarly, the step $(10\text{¢})^2 = 100\text{¢}$ is invalid.
\subsubsection{Problem 1.3 (c)}
When $a=b$, $a-b$ is equal to $0$, thus cancelling the $(a-b)$'s is invalid as it is dividing by zero.
\subsection{Problem 1.4}
While the implications going from each statement each happen to be true, this is not a valid proof as the directionality of the implicaitons should be reversed. If the proof started with $(a-b)^2 \ge 0$ and did the steps in reverse, finishing at $\frac{a+b}{2} \ge \sqrt{ab}$, then the proof would have been valid.
\subsection{Problem 1.5}
This is the \href{https://en.wikipedia.org/wiki/Unexpected_hanging_paradox}{Unexpected Hanging Paradox}, and there is no consensus to its nature. My opinion is that the ``it will be a surprise'' element of the statement is equivalent to ``the day cannot be deduced from this statement''. But, this is self-referencial, leading to a paradox in a manner similar to the \href{https://en.wikipedia.org/wiki/Liar_paradox}{Liar Paradox}.
\end{document}
